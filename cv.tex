\documentclass{tccv}
\usepackage[finnish,english]{babel}
\usepackage{fontspec} 
\usepackage[utf8x]{inputenc}
\usepackage{url}

\defaultfontfeatures{Mapping=tex-text}
\setmainfont{Fontin}

\begin{document}

\part{Joonas Lauri Jooseppi Herranen}

\section{Koulutus}
\begin{yearlist}
	\item[Teoreettinen fysiikka, \href{https://wiki.helsinki.fi/display/mathstatOpiskelu/Kokonaisuuksien+arvostelu}{arvosana} 4]{22.6.2016}
	{Filosofian maisteri}
	{Pro gradu -tutkielman arvosana: Laudatur
	\href{mailto:registrar@helsinki.fi}{Helsingin yliopisto}}
	
	\item[Teoreettinen fysiikka]{28.8.2015}
	{Luonnontieteiden kandidaatti}
	{Helsingin yliopisto}
	\item[] {2011}
	{Ylioppilas}
	{Karkkilan lukio}
\end{yearlist}

\section{Muu koulutus ja erityistaidot}
\begin{yearlist}
	\item[Fysiikka, Matematiikka, Kemia]{2016}
	{Aineenopettaja}
	{Helsingin yliopisto}
\end{yearlist}

\subsection{Ohjelmointitaidot}
\begin{factlist}
	\item{Erityistaidot}
	{Fortran, Python, \LaTeX, Matlab, Java, Shell}
	
	\item{Hyvät taidot}
	{SQL, C, HTML}
\end{factlist}

\section{Kielitaito}
\begin{factlist}
	\item{Suomi}{Äidinkieli}
	\item{Englanti}{Erinomainen}
	\item{Ruotsi}{Virkamiesruotsi}
	\item{Japani}{Perusteet}
\end{factlist}

\section{Palkinnot ja huomionosoitukset}
\begin{yearlist}
	\item{2016}
	{Pro gradu -palkinto, 500€}
	{Matemaattis-luonnontieteellinen tiedekunta, Helsingin yliopisto}
	\item{2013, 2015}
	{Opintostipendi, 950€}
	{Hämäläisten ylioppilaiden säätiö}	
	\item{2013, 2015}
	{Apuraha perustutkinto-opiskelijalle, 1000€}
	{Matematiikan ja luonnontieteiden rahasto}	
	\item{2011}
	{Kansainvälisten kemiaolympialaisten pronssimitali}
	{IChO 2011}	
\end{yearlist}

\personal
[1.7.1992, Helsinki, Suomi]
{Abraham Wetterin tie 14 C 37, 00880 Helsinki}
{+358 45 356 2399}
{joonas.herranen@helsinki.fi}
{https://www.github.com/jherrane}

\section{Työkokemus}
\begin{eventlist}	
	\item{Elokuu 2016 -- }
	{Helsingin yliopisto, Fysiikan laitos}
	{Tohtorikoulutettava}
	
	Sähkömagneettisen säteilyn siroaminen avaruuden pienhiukkasista. Siroamisen ja hiukkasen dynamiikan kytkeytymistä käsittelevän ohjelmistokehyksen kehittäminen ja testaaminen. Ohjelmistokehyksen soveltaminen polarisaatiotutkimuken avoimiin ongelmiin.
	
	\item{Syyskuu 2014 --}
	{Ursa ry}
	{Kerhonohjaaja}
	
	Kerho- ja kurssimuotoisen matematiikan ja fysiikan intensiiviopetuksen suunnittelu ja järjestäminen. Intensiiviopetuksen menetelmien kehittäminen. 
	
	\item{Kesäkuu -- Lokakuu 2015}
	{Helsingin yliopisto, Fysiikan laitos}
	{Tutkimusavustaja}
	
	Tähtienvälisen väliaineen hiukkasten asemoitumisen teoreettinen tarkastelu. Teorian mukaisen ohjelmistokehyksen kehitystyö ja pro gradu -tutkielmassa hyödynnettyjen ohjelmistojen kehitys sekä testaus.
	
	\item{Kesäkuu -- Lokakuu 2014}
	{Maanpuolustuskorkeakoulu, Sotatekniikan laitos}
	{Korkeakouluharjoittelija}
	
	Suomalaisen kriittisen infrastruktuurin tutkimus ja da\-tan\-ke\-ruun perusteella infrastruktuurin riippuvuussuhteiden ja vikasietoisuuden mallintaminen.
	
\end{eventlist}
\pagebreak

% ENGLISH 
\begin{otherlanguage}{english}

\partEng{Joonas Lauri Jooseppi Herranen}

\section{Education}
\begin{yearlist}
	\item[Theoretical physics, overall \href{https://www.helsinki.fi/en/grading}{grade} 4]{Jun 22 2016}
	{Master of Science}
	{\href{https://www.helsinki.fi/en/grading}{Grade} of Master's thesis: Laudatur
		
		\href{mailto:registrar@helsinki.fi}{University of Helsinki}}
	
	\item[Theoretical physics]{Aug 28 2015}
	{Bachelor of Science}
	{University of Helsinki}
\end{yearlist}

\section{Other education and special training}
\begin{yearlist}
	\item[Physics, Mathematics, Chemistry]{2016}
	{Subject teacher in secondary level}
	{University of Helsinki}
\end{yearlist}

\subsection{IT skills}
\begin{factlist}
	\item{Special expertise}
	{Fortran, Python, \LaTeX, Matlab, Java, Shell}
	
	\item{Good skills}
	{SQL, C, HTML}
\end{factlist}

\section{Language proficiency}
\begin{factlist}
	\item{Finnish}{Native}
	\item{English}{Fluent}
	\item{Swedish}{Bureaucratese}
	\item{Japanese}{Basics}
\end{factlist}

\section{Honors}
\begin{yearlist}
	\item{2016}
	{Pro gradu -award for exceptional Master's thesis, 500€}
	{Faculty of Science, University of Helsinki}
	\item{2013, 2015}
	{Study grant, 950€}
	{The Häme Students Foundation}	
	\item{2013, 2015}
	{Undergraduate grant, 1000€}
	{Fund for mathematics and natural sciences}	
	\item{2011}
	{Bronze medal in the International Chemistry Olympiad}
	{IChO 2011}	
\end{yearlist}

\personal
[1.7.1992, Helsinki, Suomi]
{Abraham Wetterin tie 14 C 37, 00880 Helsinki}
{+358 45 356 2399}
{joonas.herranen@helsinki.fi}
{https://www.github.com/jherrane}

\section{Vocational experience}
\begin{eventlist}	
	\item{Aug 2016 -- }
	{University of Helsinki, Department of Physics}
	{Doctoral student}
	
	The study of electromagnetic scattering from small space dust particles. Development and testing of a software framework for solving the scattering-dynamical interaction in the dust particles. Application in studying the open problems in dust polarization observations.
	
	\item{Sept 2014 --}
	{Ursa ry}
	{Science instructor}
	
	Planning, organisation and execution of intensive tuition of mathematics and physics. Development of intensive education methods.
	
	\item{Jun -- Oct 2015}
	{University of Helsinki, Department of Physics}
	{Research assistant}
	
	Theoretical review of the alignment of particles in the interstellar medium. Development of a software framework based on the theory. Development and testing of the sotware used in the Master's thesis on the subject.
	
	\item{Jun -- Oct 2014}
	{National Defense University, \hspace*{2in} \linebreak
	Department of Military Technology}
	{University trainee}
	
	Study and data collection of Finnish critical infrastructure. Modeling of the dependencies and fault tolerance of the infrastructure based on the data.
	
\end{eventlist}
\end{otherlanguage}

\end{document}

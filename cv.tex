%--------------------------------------------------------------------------------------------------%
%	The MIT License (MIT)
%
%	Copyright (c) 2019 Jan Küster
%
%	Permission is hereby granted, free of charge, to any person obtaining a copy
%	of this software and associated documentation files (the "Software"), to deal
%	in the Software without restriction, including without limitation the rights
%	to use, copy, modify, merge, publish, distribute, sublicense, and/or sell
%	copies of the Software, and to permit persons to whom the Software is
%	furnished to do so, subject to the following conditions:
%	
%	THE SOFTWARE IS PROVIDED "AS IS", WITHOUT WARRANTY OF ANY KIND, EXPRESS OR
%	IMPLIED, INCLUDING BUT NOT LIMITED TO THE WARRANTIES OF MERCHANTABILITY,
%	FITNESS FOR A PARTICULAR PURPOSE AND NONINFRINGEMENT. IN NO EVENT SHALL THE
%	AUTHORS OR COPYRIGHT HOLDERS BE LIABLE FOR ANY CLAIM, DAMAGES OR OTHER
%	LIABILITY, WHETHER IN AN ACTION OF CONTRACT, TORT OR OTHERWISE, ARISING FROM,
%	OUT OF OR IN CONNECTION WITH THE SOFTWARE OR THE USE OR OTHER DEALINGS IN
%	THE SOFTWARE.
%	
%
%--------------------------------------------------------------------------------------------------%
\documentclass[11pt,A4]{article}	
\usepackage[utf8]{inputenc}		
\usepackage{xstring, xifthen}

\usepackage{CormorantGaramond}

\usepackage[T1]{fontenc}
\usepackage{moresize}
\usepackage{fontawesome}

\newcommand{\vcenteredinclude}[1]{\begingroup
	\setbox0=\hbox{\includegraphics{#1}}%
	\parbox{\wd0}{\box0}\endgroup}

\newcommand*{\vcenteredhbox}[1]{\begingroup
	\setbox0=\hbox{#1}\parbox{\wd0}{\box0}\endgroup}

\newcommand{\icon}[3] { 							
	\makebox(#2, #2){\textcolor{maincol}{\csname fa#1\endcsname}}
}	

\newcommand{\icontext}[4]{ 						
	\vcenteredhbox{\icon{#1}{#2}{#3}}  \hspace{2pt}  \parbox{0.9\mpwidth}{\textcolor{#4}{#3}}
}

\newcommand{\iconhref}[4]{ 						
	\vcenteredhbox{\icon{#1}{#2}{black}}  \hspace{2pt} \href{#4}{{#3}}
}

\newcommand{\iconemail}[5]{ 						
	\vcenteredhbox{\icon{#1}{#2}{#5}}  \hspace{2pt} \href{mailto:#4}{\textcolor{#5}{#3}}
}

%----------------------------------------------------------------------------------------
%	PAGE LAYOUT  DEFINITIONS
%----------------------------------------------------------------------------------------

% page outer frames (debug-only)
%\usepackage{showframe}		

% we use paracol to display breakable two columns
\usepackage{paracol}

\usepackage[a4paper]{geometry}
\geometry{top=1cm, bottom=1cm, left=1cm, right=1cm}

\usepackage{fancyhdr}
\pagestyle{empty}

% space between header and content
% \setlength{\headheight}{0pt}

% indentation is zero
\setlength{\parindent}{0mm}

\usepackage{array}
\newcolumntype{x}[1]{%
	>{\raggedleft\hspace{0pt}}p{#1}}%

\usepackage{graphicx}
% use this for floating figures
% \usepackage{wrapfig}
% \usepackage{float}
% \floatstyle{boxed} 
% \restylefloat{figure}

%for drawing graphics		
\usepackage{tikz}				
\usetikzlibrary{shapes, backgrounds, mindmap, trees}

\usepackage{transparent}
\usepackage{color}
% primary color
\definecolor{maincol}{RGB}{ 169, 10, 10 }
% accent color, secondary
\definecolor{accentcol}{RGB}{ 250, 150, 10 }
% dark color
%\definecolor{darkcol}{RGB}{ 10, 100, 100 }
\definecolor{darkcol}{rgb}{0.23, 0.27, 0.29}
% light color
\definecolor{lightcol}{RGB}{245,245,245}

\usepackage[hidelinks]{hyperref}
\hypersetup{
	colorlinks=true,
	linkcolor=blue,
	filecolor=magenta,      
	urlcolor=darkcol,
}

% returns minipage width minus two times \fboxsep
% to keep padding included in width calculations
% can also be used for other boxes / environments
\newcommand{\mpwidth}{\linewidth-\fboxsep-\fboxsep}

%============================================================================%
%
%	CV COMMANDS
%
%============================================================================%
\newcommand{\cvlist}[1] {
	\begin{itemize}{#1}\end{itemize}
}

\newcommand{\cvtext}[1] {
	\begin{tabular*}{1\mpwidth}{p{0.98\mpwidth}}
		\parbox{1\mpwidth}{#1}
	\end{tabular*}
}

\newcommand{\cvsection}[1] {
	\vspace{14pt}
	\cvtext{
		\textbf{\LARGE{\textcolor{darkcol}{{#1}}}}\\[-4pt]
		\textcolor{maincol}{ \rule{0.1\textwidth}{2pt} } \\
	}
}

\newcommand{\cvskill}[3] {
	\begin{tabular*}{1\mpwidth}{p{0.72\mpwidth}  r}
		\textcolor{black}{\textbf{#1}} & \textcolor{maincol}{#2}\\
	\end{tabular*}%
	
	\hspace{4pt}
	\begin{tikzpicture}[scale=1,rounded corners=2pt,very thin]
	\fill [lightcol] (0,0) rectangle (1\mpwidth, 0.15);
	\fill [maincol] (0,0) rectangle (#3\mpwidth, 0.15);
	\end{tikzpicture}%
}

\newcommand{\cvevent}[4] {
	\parbox{\mpwidth}{
		\begin{tabular*}{1\mpwidth}{p{0.72\mpwidth}  r}
			\textcolor{black}{\textbf{#2}} & \colorbox{maincol}{\makebox[0.25\mpwidth]{\textcolor{white}{#1}}} \\
			\textcolor{maincol}{\textbf{#3}} & \\
		\end{tabular*}\\[8pt]
		
		\ifthenelse{\isempty{#4}}{}{
			\cvtext{#4}\\
		}
	}
	
	\vspace{14pt}
}

\newcommand{\cvmetaevent}[4] {
	\textcolor{maincol} {\cvtext{\textbf{\begin{flushleft}#1\end{flushleft}}}}
	
	\ifthenelse{\isempty{#2}}{}{
		\textcolor{darkcol} {\cvtext{\textbf{#2}}\\[-12pt] }
	}
	
	\ifthenelse{\isempty{#3}}{}{
		\cvtext{{ \textcolor{darkcol} {#3} }}\\[-7.5pt]
	}
	
	\cvtext{#4}\\[10pt]
}

\newcommand{\cvqrcode}[1] {
	\begin{center}
		\includegraphics[width={#1}\mpwidth]{qrcode}
	\end{center}
}

\begin{document}
	\columnratio{0.31}
	\setlength{\columnsep}{2.2em}
	\setlength{\columnseprule}{4pt}
	\colseprulecolor{lightcol}
	\begin{paracol}{2}
		\begin{leftcolumn}
			
			\cvsection{Henkilötiedot}
			
			\vspace*{-4pt}
			\icontext{Male}{12}{\small 1. 7. 1992, Helsinki, Suomi}{black}\\[3pt]
			\icontext{MapMarker}{12}{\small Abraham Wetterin tie 14 C 37\\00880 Helsinki, Finland}{black}\\[3pt]
			\icontext{MobilePhone}{12}{\small +358 45 356 2399}{black}\\[3pt]
			\iconemail{Envelope}{12}{\small joonas.herranen@iki.fi}{joonas.herranen@iki.fi}{black}\\[3pt]
			\iconhref{Github}{12}{\small github.com/jherrane}{https://github.com/jherrane}{}\\[-15pt]
			
			\cvsection{Koulutus}
			\\[-20pt]
			\cvmetaevent
			{2016 -- 2020}
			{FT, Tähtitiede}
			{Helsingin yliopisto}
			{\footnotesize Tutkimus professori Karri Muinosen alaisuudessa.
			}\\[-20pt]
			\cvmetaevent
			{2015 -- 2016}
			{FM, teoreettinen fysiikka}
			{Helsingin yliopisto}
			{\footnotesize \href{https://studies.helsinki.fi/instructions/article/grades-and-assessment}{Kokonaisarvosana 4}
				
				Pro gradu -tutkielman arvosana: 
				
				Laudatur
			}
			\\[-15pt]
			\cvmetaevent
			{2012 -- 2015}
			{LK, teoreettinen fysiikka}
			{}
			{Helsingin yliopisto}\\[-40pt]
			
			\cvmetaevent
			{2015 -- 2019}
			{Aineenopettaja}
			{Helsingin yliopisto}
			{\footnotesize Lukion ja yläkoulun fysiikka, matematiikka, kemia ja tietotekniikka.
			}\\[-30pt]
			
			\cvsection{Taidot}\vspace{-5pt}
			
			\cvskill{\small Fortran, Python,
				Matlab} {>5 vuotta} {1} \\[-4pt]
			
			\cvskill{\small Linux, Git, \LaTeX} {>4 vuotta} {0.95} \\[-4pt]
			
			\cvskill{\small Html/CSS, SQL} {>3 vuotta} {0.66} \\[-4pt]
			
			\vspace*{-15pt}
			\cvsection{Kielitaito}\vspace*{-10pt}
			
			\cvskill{Suomi}{\hspace{-5pt}Äidinkieli}{} \\[-5pt]
			
			\cvskill{Englanti}{\hspace{-20pt}Erinomainen}{0.95} \\[-5pt]
			
			\cvskill{Ruotsi}{\hspace{-33pt}Virkamiesruotsi}{0.5} \\[-5pt]
			
			\cvskill{Japani}{\hspace{-40pt}Keskustelutaidot}{0.4} \\[-5pt]
			
			%			\mbox{} % hotfix to place qrcode on the bottom when there are not other elements
			
		\end{leftcolumn}
		\begin{rightcolumn}
			
			\fcolorbox{white}{darkcol}{\begin{minipage}[c][2.cm][c]{1\mpwidth}
					\begin {center}
					\huge{ \textbf{ \textcolor{white}{ \uppercase{Joonas Herranen} } } } \\[4pt]
					\large{ \textcolor{white} {Ansioluettelo} }
					\end {center}
			\end{minipage}} 
			
			\cvsection{Työkokemus}
			
			\cvevent{Syys 2020 --}{Waveguide designer}{Dispelix Oy, Espoo, Finland}{Numeerinen valonohjainten mallinnus AR-linssiteknologian kehitystyössä.}\vspace*{-15pt}
			
			\cvevent{2016 -- 2020}{Tohtorikoulutettava}{Helsingin yliopisto}{Sähkömagneettisen säteilyn siroaminen avaruuden pienhiukkasista. Siroamisen ja hiukkasen dynamiikan kytkeytymistä käsittelevän ohjelmistokehyksen kehittäminen ja testaaminen. Ohjelmistokehyksen soveltaminen polarisaatiotutkimuken avoimiin ongelmiin.}
			
			\cvevent{2014 -- 2018}{Kerhonohjaaja}{Ursa ry}{Kerho- ja kurssimuotoisen matematiikan ja fysiikan intensiiviopetuksen suunnittelu ja järjestäminen. Intensiiviopetuksen menetelmien kehittäminen.}
			
			\cvevent{Kesä -- Loka 2015}{Tutkimusavustaja}{Helsingin yliopisto, Fysiikan laitos}{Tähtienvälisen väliaineen hiukkasten asemoitumisen teoreettinen tarkastelu. Teorian mukaisen ohjelmistokehyksen kehitystyö ja pro gradu -tutkielmassa hyödynnettyjen ohjelmistojen kehitys sekä testaus.}
			
			\cvevent{Kesä -- Loka 2014}{Korkeakouluharjoittelija}{Maanpuolustuskorkeakoulu, Sotatekniikan laitos}{Suomalaisen kriittisen infrastruktuurin tutkimus ja da\-tan\-ke\-ruun perusteella infrastruktuurin riippuvuussuhteiden ja vikasietoisuuden mallintaminen.}
			
			\vspace*{-15pt}
			\cvsection{Palkinnot ja huomionosoitukset}
			
			\cvevent
			{2016}
			{Pro Gradu -palkinto}
			{Matemaattis-luonnontieteellinen tiedekunta, Helsingin yliopisto}
			{}
			\vspace{-10pt}
			\cvevent
			{2015, 2013}
			{Opintostipendi}
			{Hämäläisten ylioppilaiden säätiö}
			{}
			\vspace{-10pt}
			\cvevent
			{2015, 2013}
			{Apuraha perustutkinto-opiskelijalle}
			{Matematiikan ja luonnontieteiden rahasto}
			{}
			\vspace{-10pt}
			\cvevent
			{2011}
			{Kansainvälisten kemiaolympialaisten pronssimitali}
			{IChO 2011}
			{}
			
			
			% hotfixes to create fake-space to ensure the whole height is used
			%			\mbox{}
			%			\vfill
			%			\mbox{}
			%			\vfill
			%			\mbox{}
			%			\vfill
			%			\mbox{}
		\end{rightcolumn}
	\end{paracol}
	
\end{document}


1